\documentclass[11pt, a4paper]{article}

\usepackage{fontspec}
\usepackage{geometry}
\usepackage{fancyhdr}
\usepackage[hidelinks]{hyperref}
\usepackage[normalem]{ulem}
\usepackage{multicol}

\geometry{
  top=3cm,
  bottom=3cm,
  left=3cm,
  right=3cm,
  marginparsep=4pt,
  marginparwidth=1cm
}

\renewcommand{\headrulewidth}{0pt}
\pagestyle{fancyplain}
\fancyhf{}
\lfoot{}
\rfoot{}

\setlength{\parindent}{0pt}
\setlength{\parskip}{0pt}

\usepackage{xunicode}
\defaultfontfeatures{Mapping=tex-text}

\begin{document}

\begin{center}
\textbf{DSST289: Introduction to Data Science --- Taylor Arnold --- Fall 2025}
\end{center}

\vspace{0.5cm}

\textbf{Website}: \texttt{https://taylor-arnold.github.io/courses/dsst289-f25}

\bigskip

\textbf{Topics:}
Methods for collecting, manipulating, visualizing, exploring, and presenting
data.

\bigskip

\textbf{Format:}
The semester is broken into four units, with an exam at the end of each. The
last week of the semester is dedicated to a final project. Class meetings are
hands-on and interactive. Please bring a computer, pencil, and something to
write on to each class. Class materials can be found on the course website.

\bigskip

\textbf{Reading \& Class Form:}
Most class meetings will have a short reading posted on our website.
We will fill out a brief class form each meeting, which is used for attendance
and to pledge that you have done the reading. A class form grade will be given
by subtracting 5\% from your grade of 100\% for each missing form after the first
two (in other words, you have two excused missing forms).

\bigskip

\textbf{Software:}
We will be programming in the open-source Python programming language. 
Classwork can be run through Google's Colab environment in a web browser using
a University email login. Other hardware and software may be used for some data collection tasks; all of will be provided or freely accessible.

\bigskip

\textbf{Exams:}
There are four midterm exams. Each will have an in-class component. Some exams,
particularly the final one, may be an additional take-home component due on
the day of the exam. A list of topics for each exam will be posted on the course
website.

\bigskip

\textbf{Final Project:}
A final project will be due during the last week of class. The project will
focus on finding or creating a new dataset and applying the techniques learned
throughout the class to the analysis of it. Detailed instructions will
be posted following fall break.

\bigskip

\textbf{Getting Help:}
We will usually have time in class to answer questions about the course
material. If questions arise outside of class, please feel free to send
these by email. I am happy to schedule office hours for any extended questions
or personal concerns. Because I know everyone has a busy schedule, I offer
office hours primarily by appointment.

\bigskip

\textbf{Final Grades:}
The three exams, final project, and class form grades are averaged (i.e., each
contributes 1/6 of your grade). A letter grade is assigned as follows:
             A (93--100), A- (90--92),
B+ (87--89), B (83--86),  B- (80--82),
C+ (77--79), C (73--76),  C- (70--72), and F (0--69).

\end{document}
