\documentclass[11pt, a4paper]{article}

\usepackage{fontspec}
\usepackage{geometry}
\usepackage{fancyhdr}
\usepackage[hidelinks]{hyperref}
\usepackage[normalem]{ulem}
\usepackage{multicol}

\geometry{
  top=3cm,
  bottom=3cm,
  left=3cm,
  right=3cm,
  marginparsep=4pt,
  marginparwidth=1cm
}

\renewcommand{\headrulewidth}{0pt}
\pagestyle{fancyplain}
\fancyhf{}
\lfoot{}
\rfoot{}

\setlength{\parindent}{0pt}
\setlength{\parskip}{0pt}

\usepackage{xunicode}
\defaultfontfeatures{Mapping=tex-text}

\begin{document}

\begin{center}
\textbf{DSST289: Introduction to Data Science --- Taylor Arnold --- Fall 2025}
\end{center}

\vspace{0.5cm}

\textbf{Website}: \texttt{https://taylor-arnold.github.io/courses/dsst289-f25}

\bigskip

\textbf{Topics:}
Methods for collecting, manipulating, visualizing, exploring, and presenting
data.

\bigskip

\textbf{Format:}
Class meetings are hands-on and interactive. We will start with a brief
overview of new material and then proceed to programming or data
collection. Bring a computer, pen/pencil, and something to write on to each class.
Most classes will have a brief reading or task to do as preparation. Course
materials and assignments can be found on the website above. 

\bigskip

\textbf{Software:}
We will be programming primarily in the open-source Python programming language.
Classwork can be run through Google's Colab environment in a web browser using
a University email login. 

\bigskip

\textbf{Quizzes:}
There will be a written quiz at the start of each Wednesday's class, from the
second week through Thanksgiving Break. The subject of each quiz will be clearly
given in a study guide posted on the website. Quizzes are cumulative and 
questions may be drawn from any previous study guide. Each will be graded on
a pass/fail basis, with a passing grade requiring a minimum of 75\% of
questions answered correctly. Make-up quizzes are only possible for official
university excused absences.

\bigskip

\textbf{Class Form:}
We will fill out a brief class form each meeting, which is used for attendance
and to pledge that you have done the reading. If late or absent, please fill out
the form sometime after the class meeting. 

\bigskip

\textbf{Final Project:}
A final project will be due during the last week of class. The project will
focus on finding or creating a new dataset and applying the techniques learned
throughout the class to the analysis of it. Detailed instructions will
be posted following fall break. Projects will receive a pass/fail grade based on
the rubric distributed along with the assignment.

\bigskip

\textbf{Getting Help:}
We will usually have time in class to answer questions about the course
material. If questions arise outside of class, please feel free to send
these by email. I am happy to schedule office hours for
any extended questions or personal concerns. Because I know everyone has a
busy schedule, I offer office hours primarily by appointment.

\bigskip

\textbf{Final Grades:}
The following guarantees the minimum requirements for each letter grade:
\begin{description}
   \setlength\itemsep{0em}
  \item \textbf{A} \textrightarrow\, pass final project, pass 10/12 quizzes,
  miss no more than 3 classes
  \item \textbf{B} \textrightarrow\, pass final project, pass 8/12 quizzes,
  miss no more than 4 classes
  \item \textbf{C} \textrightarrow\, pass final project, pass 6/12 quizzes,
  miss no more than 6 classes
\end{description}
Absenses during the first week (add/drop) are not considered in these scores.
Intermediate grades are assigned at the instructor's discretion taking into 
account all completed work. No grades of \textbf{C-} or below will be given.

\end{document}
